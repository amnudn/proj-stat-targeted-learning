
% Created 2018-08-01 Wed 12:57
% Intended LaTeX compiler: pdflatex
\documentclass{article}
               \usepackage{listings}
               \usepackage{color}
               \usepackage{tocloft}

               \usepackage{times}
%\usepackage[cmbold]{mathtime}
\usepackage{bm}
\usepackage{natbib}



% --------- OWN START

\newenvironment{proof}{\textit{Proof.}}{\hfill$\square$}

\usepackage{hyphenat}
\usepackage{mathtools}
\hyphenation{pa-ra-me-trized}
\usepackage{bbold}
\usepackage{bbm}
\usepackage[T1]{fontenc}
\usepackage{amsmath}
\usepackage{amssymb}
\usepackage{prodint}
\usepackage{color}

               
\usepackage{amsmath}
\usepackage{array}
\usepackage[T1]{fontenc}
\usepackage{natbib} 
\bibliographystyle{abbrvnat}
\setcitestyle{authoryear,open={(},close={)}}
\usepackage{authblk} 
\usepackage{mathrsfs}
\usepackage{enumitem} 
\newcommand*{\qeda}{\hfill\ensuremath{\tiny\CIRCLE}} 
\usepackage[small]{titlesec}
\titleformat{\subsection}[runin]{\bfseries}{\thesubsection}{1em}{}
\titleformat{\subsubsection}[runin]
{\normalfont\bfseries}{\thesubsubsection}{1em}{}
%{\normalfont\itshape}{\normalfont\thesubsubsection}{1em}{}
\usepackage{adjustbox}
\usepackage{prodint}
\usepackage{booktabs}
\usepackage{bm}
\usepackage{bbold}
%\setlength\parindent{0pt}
\usepackage{wasysym}
\usepackage[margin=1.3in]{geometry}
\setcounter{secnumdepth}{4}

\newcommand{\cadlag}{c\`{a}dl\`{a}g }
\newcommand{\Cadlag}{C\`{a}dl\`{a}g }
\newcommand{\EE}{\mathbb{E}}
\newcommand{\one}{1}
\newcommand{\eps}{\varepsilon}
\newcommand{\VV}{V}
\newcommand{\PP}{\mbox{P}}
\newcommand{\norm}{\mathcal{N}}
\newcommand{\lag}{N}
\newcommand{\str}{S}
\newcommand{\smin}{s^{\min}}
\newcommand{\smax}{s^{\max}}
\newcommand{\styp}{s^{*}}
\newcommand{\period}{[a,b]}
\newcommand{\periodK}{\ensuremath{[T_k,T_{k+1})}}
\newcommand{\K}{K}
\newcommand{\kk}{k}
\newcommand{\D}{D}
\newcommand{\B}{B}
\newcommand{\F}{\mathcal{F}}
\newcommand{\E}{E}
\newcommand{\XX}{X}
\newcommand{\QQ}{Q}
\newcommand{\Ru}{R}
\newcommand{\GG}{G}
\newcommand{\T}{T}
\newcommand{\st}{s}
\newcommand{\Nn}{N}
\newcommand{\C}{C}
\newcommand{\uu}{u}
\newcommand{\vv}{v}
\newcommand{\zz}{z}
\newcommand{\ww}{w}
\newcommand{\M}{M}
\newcommand{\I}{I}
\newcommand{\RR}{R}
\newcommand{\R}{\mathbb{R}}
\newcommand{\N}{\mathbb{N}}
\newcommand{\1}{\mathbb{1}}
\newcommand{\ses}{SES}
\newcommand{\Y}{Y}
\newcommand{\htt}{h}
\newcommand{\dtt}{d}
\newcommand{\tM}{M}
\newcommand{\ta}{\tau_0}
\newcommand{\tb}{\tau_1}
\newcommand{\intd}{\mathcal{D}}
\newcommand{\nW}{n_W}
\newcommand{\nL}{n_L}
\newcommand{\tmax}{\tau}
\newcommand{\A}{A}
\newcommand{\LL}{L}
\newcommand{\W}{W}
\newcommand{\X}{X}
\newcommand{\U}{U}
\newcommand{\Z}{Z}
\newcommand{\btheta}{\bm{\theta}}
\newcommand{\Pvar}{W}
\newcommand{\pvar}{p}
\newcommand{\logit}{\text{logit}}
\newcommand{\expit}{\text{expit}}
\newcommand{\argmax}{\text{argmax}}
\newcommand{\argmin}{\text{argmin}}
\newcommand{\cupdot}{\mathbin{\mathaccent\cdot\cup}}
\newcommand{\qed}{\hfill\(\square\)}
\newcommand\independent{\protect\mathpalette{\protect\independenT}{\perp}}\def\independenT#1#2{\mathrel{\rlap{$#1#2$}\mkern2mu{#1#2}}}
\newtheorem{thm}{Theorem}[section]
\newtheorem{claim}{Claim}[section]
\newtheorem{defi}{Definition}
\newtheorem{lemma}[thm]{Lemma}
\newtheorem{example}[thm]{Example}
\newtheorem{obs}{Observation}
\newtheorem{choice}{Choice}
\newtheorem{drug}{Drug type}
\renewcommand\thedrug{\Roman{drug}}
\newtheorem{remark}{Remark}
\newtheorem{assumption}{Assumption}
\newtheorem{inclusion}{Inclusion criterion}
\newtheorem{proxy}{Proxy}
\usepackage{tikz}
\usetikzlibrary{arrows}
\tikzset{every picture/.style=remember picture}
\newcommand{\mathnode}[1]{\mathord{\tikz[baseline=(#1.base), inner sep = 0pt]{\node (#1) {$#1$};}}}

\lstset{
keywordstyle=\color{blue},
commentstyle=\color{red},stringstyle=\color[rgb]{0,.5,0},
literate={~}{$\sim$}{1},
basicstyle=\ttfamily\small,
columns=fullflexible,
breaklines=true,
breakatwhitespace=false,
numbers=left,
numberstyle=\ttfamily\tiny\color{gray},
stepnumber=1,
numbersep=10pt,
backgroundcolor=\color{white},
tabsize=4,
keepspaces=true,
showspaces=false,
showstringspaces=false,
xleftmargin=.23in,
frame=single,
basewidth={0.5em,0.4em},
}

\usepackage{authblk}
\usepackage[utf8]{inputenc}
\usepackage[T1]{fontenc}
\usepackage{graphicx}
\usepackage{grffile}
\usepackage{longtable}
\usepackage{wrapfig}
\usepackage{rotating}
\usepackage[normalem]{ulem}
\usepackage{amsmath}
\usepackage{textcomp}
\usepackage{amssymb}
\usepackage{capt-of}
\usepackage{hyperref}


\title{\textsc{Targeted learning for causal effect estimation} \\[0.5em]
%{  \large or... } \\
%  \textsc{Influence function based nonparametric efficient estimation of causal effect } \\[0.5em] 
\large Project instructors:\\[-0.8em]}
%\author{\large Project instructors: \normalsize \\ Helene Rytgaard (hely@sund.ku.dk) \\
% and \\ Anders  Munch (a.munch@sund.ku.dk)}
\author[1]{\normalsize Helene Rytgaard (hely@sund.ku.dk) }
\author[1]{Anders Munch (a.munch@sund.ku.dk)}
\affil[1]{Section of Biostatistics, University of Copenhagen}
\date{}                     %% if you don't need date to appear
\setcounter{Maxaffil}{0}
\renewcommand\Affilfont{\itshape\small}


\begin{document}

\maketitle


% \color{red} Right now I wrote out like it has to be targeted
% learning; but could just be semiparametric efficient
% estimation... :) \color{black} \\

%\tableofcontents





In this project we consider causal effect estimation in semiparametric
settings where the target parameter is low-dimensional but estimation
has to deal with potentially high-dimensional nuisance parameters
often taking the form of a regression function. In these settings,
efficient influence function based estimation provides a popular basis
for combining machine learning techniques with valid statistical
inference. We will consider and investigate these methods for
estimation of the average treatment effect (ATE) based on
observational (i.e., non-randomized) data.


\section{The problem to be studied}


In randomized trials, where trial participants are enrolled and
randomized to a treatment or a placebo arm, the randomization ensures
that the differences in outcome occur as a result of treatment
differences only. Randomized trials are often considered the golden
standard for causal inference, but may not always be feasible or
ethical. In this project, the question we want to address is whether
having a planned cesarian section (intended cesarian section) among
women who gave birth twice changes the risk of postpartum haemorrhage
during the second delivery. For this question, performing a randomized
trial would not be ethical% , and it is thus of high interest to turn to
% observational data for causal effect estimation
. With tools from
causal inference and semiparametric efficiency theory, we can
formulate and optimally estimate causal effects from observational
data% as
% effects we would had seen had we conducted a randomized trial
. \\

As part of the project, you will:
\begin{itemize}
\item[\(*\)] Understand key concepts from semiparametric efficiency
  theory and the bias-correction abilities of influence function based
  estimation.
\item[\(*\)] Translate a real-world data application into a
  mathematical and statistical formulation of the estimation problem
  that needs to be solved. 
\item[\(*\)] Assess model misspecification and estimation performance
  via simulations in \verb+R+.
\item[\(*\)] Conduct an analysis of the Danish Birth Registry data in
  \verb+R+ for average treatment effect estimation.
\end{itemize}






\section{Targeted learning}


The targeted learning framework has been developed as a general
template for combining machine learning and causal inference. In this
project we consider a particular data structure as follows. Suppose we
observed an iid sample \(O_1,\ldots,O_n\), \(n\in\mathbb{N}\) of a
random variable \(O \in\mathcal{O}\) distributed according to an
unknown distribution function \(P_0\) belonging to a statistical model
\(\mathcal{M}\). Each observation consists of \(O= (X, A, Y)\) where
\(X\in \R^d\) are covariates, \(A\in \lbrace 0,1\rbrace\) is a binary
exposure and \(Y\in\lbrace 0, 1\rbrace\) is a binary outcome
variable. The target parameter can be written as functional
\(\Psi \, : \, \mathcal{M} \rightarrow \R\) of distributions
\(P\in\mathcal{M}\), for our purposes defined as
\begin{align}
  \Psi( P ) = \EE_P \big[ \EE_P[ Y \mid A=1, X] - \EE[ Y\mid A=0,X]\big],
  \label{eq:Psi:ate}
\end{align}
where \(\EE_P[\cdot]\) denotes the expectation operator under the distribution \(P\in \mathcal{M}\).
The parameter defined by \eqref{eq:Psi:ate} is commonly referred to as the \textit{average treatment
  effect} (ATE); under structural assumptions, the ATE can be interpreted causally
\citep{hernanrobins}. A straightforward estimator of the ATE would be
\begin{equation*}
  \frac{1}{n} \sum_{i=1}^{n}
  \left\{
    \hat{f}_n(1, X_i) - \hat{f}_n(0, X_i)
  \right\},
\end{equation*}
where $\hat{f}_n$ denotes some estimator of the conditional
expectation $\EE[Y \mid A, X]$. Note that the task of estimating
$\EE[Y \mid A, X]$ is a standard classification problem, and hence we
have a large collection of well-studied estimators of at our disposal,
e.g., logistic regression, random forests or neural networks. Perhaps
surprisingly, nice performance for estimation of $\EE[Y \mid A, X]$
does not necessarily translate into nice performance for estimation of
the target parameter $\Psi(P)$. Basing the estimation procedure on the
so-called efficient influence function, on the other hand, shifts the
performance optimization to the target parameter specifically. This
requires estimating both the exposure distribution \(P(A=a \mid X)\)
as well as the outcome regression above (potentially using machine
learning), but then provides the basis for asymptotic linearity and
efficiency of the resulting estimator and thus inference based on the
limiting normal distribution. Thus, estimation based on influence
functions allow us to use flexible estimators for \(P_0\), to ensure
optimal asymptotic behavior of the estimator without strict parametric
assumptions.

% To estimate \(\psi_0 = \Psi(P_0)\), one needs to estimate \(P_0\), or relevant
% components of \(P_0\). With an estimator \(\hat{P}_n\), one obtains a so-called plug-in estimator
% for \(\psi_0\) as \(\hat{\Psi}_n = \Psi( \hat{P}_n)\). To estimate the ATE, one needs specifically
% to estimate the conditional expectations \(\EE_{P_0}[ Y \mid A=1, X]\) and
% \(\EE_{P_0}[ Y \mid A=0, X]\). These are simple binary regressions; thus, we can apply logistic
% regression, or any machine learning algorithm working with binary outcomes. After obtaining such
% estimators, an estimator for the target parameter can be obtained by taking the empirical average of
% the difference. However,

In this project we focus on the specific problem of estimating the ATE, but the field of targeted
learning can more broadly be seen as one approach to combine the advantages of machine learning
(flexible and data-adaptive models with few assumptions) with the goal of more traditional inference
based statistics, which seeks to establish valid confidence intervals for interpretable parameters.
This is an important general problem to address if we want to utilize the advances of machine
learning to answer scientifically interesting questions.

\iffalse Suppose we observe an iid sample \(O_1,\ldots,O_n\),
\(n\in\mathbb{N}\) of a random variable \(O \in\mathcal{O}\)
distributed according to an unknown distribution function \(P_0\)
belonging to a statistical model \(\mathcal{M}\). The target parameter
can be written as functional
\(\Psi \, : \, \mathcal{M} \rightarrow \R\) of distributions
\(P\in\mathcal{M}\). To estimate \(\psi_0 = \Psi(P_0)\), we need to
estimate \(P_0\), or relevant components of \(P_0\).  With an
estimator \(\hat{P}_n\), one obtains a so-called plug-in estimator for
\(\psi_0\) as \(\hat{\Psi}_n = \Psi( \hat{P}_n)\).

An estimator \(\hat{\Psi}_n\) is said to be asymptotically linear
estimator if there exists a function
\(\phi(P_0) \, : \, \mathcal{O}\rightarrow \R\) with mean zero and
finite variance under \(P_0\) such that: 
\begin{align}
  \sqrt{n} \big( \hat{\Psi}_n - \psi_0 \big) =  \frac{1}{\sqrt{n}} \sum_{i=1}^n
  \, \phi(P_0) (O_i)  + R_n ,
  \label{eq:asymptotic:linearity}
\end{align}
where \(R_n \overset{P}{\rightarrow} 0\) as \(n\rightarrow\infty\).
When an estimator can be written on the form in
\eqref{eq:asymptotic:linearity}, \(\phi\) is called the
\textit{influence function} of \(\hat{\Psi}_n\). A straightforward
consequence of \eqref{eq:asymptotic:linearity} is the asymptotic
normality
\begin{align*}  
  \sqrt{n} \big( \hat{\Psi}_n - \psi_0 \big) \overset{\mathcal{D}}{\rightarrow} N(0, P_0 \phi(P_0)^2),
\end{align*} 
which can be used to provide asymptotic inference for the estimator
\(\hat{\Psi}_n\). A key result of semiparametric efficiency theory
\citep{bickel1993efficient} yields that one particular influence
function, called the \textit{efficient influence function} has the
lowest variance among all influence functions and thus characterizes
the best variance among all asymptotically linear estimators. The main
idea of targeted learning is to construct estimators \(\hat{\Psi}_n\)
that are asymptotically linear with influence function equal to the
efficient influence function. \fi





\section{Data}



In the project we will work with a random subset of the data described
in \cite{wikkelso2014prediction}. The dataset contains data from 48272
Danish women who gave birth twice. The main outcome is a binary
variable called \verb+PPHbin+ which indicates if the woman had a
postpartum haemorrhage (to haemorrhage means to bleed very heavily)
during the second delivery.
%
% Analyzing the section delivery has the advantage that we can use
% information from the first delivery to predict the decision to have
% a planned cesarian section and to predict the outcome.
%
The data are from the registries and have an observational
character. In particular, note that the decision to plan a cesarian
section was not randomized. With causal tools we want to analyze the
causal effect of a planned cesarian section on the risk of postpartum
haemorrhage, i.e., the effect that one would have observed in a
hypothetical study which randomizes women to either intended cesarian
section or intended vaginal birth.

\section{What you need to know before the course}

It is not required that you know semiparametric efficiency theory nor
causal inference beforehand, but you must be able to work comfortably
with:

\begin{itemize}
\item[\(*\)] Derivatives of functions, directional derivative, chain rule.
\item[\(*\)] Probability distributions, conditional distributions, empirical
  distribution
\item[\(*\)] Central limit theorem
\item[\(*\)] Regression models, including logistic regression
\item[\(*\)] Statistical inference, parameter estimation, confidence
  intervals, p-values
\item[\(*\)] \verb+R+ programming
\end{itemize}

 
\section{Project work}

Figure \ref{fig:roadmap} illustrates the roadmap of the data analysis.

\begin{figure}[!h] % h: here, t: top, b: bund, !: ikke p�nt
\begin{center}
  \makebox[\textwidth][c]{
    \includegraphics[width=0.6\textwidth,angle=0]{./figures/project-roadmap.pdf}}
\end{center}
\caption{Roadmap for the data analysis of the project.}
\label{fig:roadmap}
\end{figure}

\section{Some literature}

There are many papers and some books about the mathematical background
of the methods we consider. A very nice review is provided by
\cite{kennedy2016semiparametric}. %Visual intuition: \cite{fisher2020visually}.
For causal notation, we follow the framework with counterfactual
variables \cite{hernanrobins}. % Targeted learning is treated quite
% generally in the two books \cite{van2011targeted} and
% \cite{van2018targeted}.
We further recommend the introduction (pp. 1--11) by
\cite{chernozhukov2018double}.
 
Other more in-depth references on semiparametric efficiency theory for
the more interested are
\cite{bickel1993efficient,van2000asymptotic,vanRobins2003unified,
  tsiatis2007semiparametric}; \citet[][Appendix A]{ van2011targeted}.

\bibliography{refs}




\end{document}















































\newpage

\subsection{Exploratory analysis, January 18}
\label{sec:empirical:results}

\newpage

\subsection{Exploratory analysis, January 18}
\label{sec:empirical:results:jan:18}

In last week's simulations, the output was often NA for small sample
sizes, and \(\Sigma_n\) was not always invertible. It turns out the NA
was due to a mixture of positivity issues and the last time-point
being smaller than \(\min_i \tilde{T}_i\) (effectively giving a
variance of zero). I have now added a regularization of \(\Sigma_n\)
when needed (adding a small number -- \(0.05\) -- to the diagonal). I also
discovered that my checking of the supremum norm for the weighting by
\(\Sigma_n\) with continued updating was too strict --- I had by
mistake divided the right hand by \(\sqrt{d}\) an extra time.

Right now I focus on small number of simulation repetitions
(\(M=100\)).

For the setting \textbf{without competing risks}: \(M=100\) simulation
repetitions; average treatment effect on the survival curve across
\(t \) in a grid of thirty time-points; and sample sizes
\(n \in \lbrace 100, 200, 300, 400, 500,\)
\(600, 700, 800\rbrace\).



For the setting \textbf{with competing risks}: \(M=100\) simulation
repetitions; average treatment effect on the three subdistributions
simultaneously across \(t_0 \in \lbrace 0.45, 0.5, 1 \rbrace\); and
sample sizes \(n \in \lbrace 100, 200, 300, 400,\) \( 500\rbrace\). \\

Overall, Choices \ref{ex:2} and \ref{ex:3} seem to do quite well; they
are both faster and more robust. \\

When checking \eqref{eq:check:max:m}, this seems that this is
accomplished by weighting by \(\Sigma_n\) and continuing iterations
until
\begin{align*}
  \Vert  \mathbb{P}_n D^*(\hat{P}^*_n) \Vert 
  \le \frac{1}{\sqrt{n}\log n} .
\end{align*}
%Alternatively, we should check \eqref{eq:alternative:check}? 

\newpage

\subsection{Exploratory analysis, January 12}
\label{sec:empirical:results:jan:12}

We are interested in investigating:
\begin{enumerate}
\item When is what norm (c.f., Examples \ref{ex:1}--\ref{ex:3} of
  Section \ref{sec:hilbert:space:norms}) best?
\item[\(*\)] I have tried the following different settings:
  \begin{enumerate}
  \item targeting the survival curve across
    \(t_0 \in \lbrace 0.49, 0.5, 0.51, 0.8, 1.2, 1.5\rbrace\) for
    varying sample sizes
    (\(n \in \lbrace 200, 300, 500, 700, 800, 1000\rbrace\)) and
    \(M=500\) repetitions.
  \item targeting the survival curve across 30 equidistant time-points
    for varying sample sizes
    (\(n \in \lbrace 100, 200, 300, 400, 500, 600 \rbrace\)), but only
    and \(M=10\) repetitions.
  \item targeting the survival curve across 30 equidistant time-points
    for varying (small) sample sizes
    (\(n \in \lbrace 100, 150, 200, 250\rbrace\)) and \(M=500\)
    repetitions.
  \item targeting three subdistributions across three time-points
    \(t_0 \in \lbrace 0.45, 0.5, 1 \rbrace\) for sample size
    \(n = 1000\) and \(M=100\) simulation repetitions. \textit{(I want
      to repeat this for smaller sample sizes as well).}
  \end{enumerate}
\item Estimation of the survival curve across a nice grid of
  time-points, while tracking the supremum norm.
\item[\(*\)] For this, I have tracked the supremum norm while running
  (b) and (c) above. \\
\end{enumerate}

\textit{What can we learn?}

\begin{itemize}
\item Coverage is good across all simulation repetitions, both for
  survival and competing risks setting.
\item One-step TMLE (obviously) produces a monotone survival curve,
  whereas iterative TMLE does not always (worse for smaller sample
  sizes).
\item For (a), one-step TMLE seems to win over iterative TMLE in terms
  of relative mse when sample size is small. 
\item For (a) and (b), norms weighted seem to converge much
  faster. Particularly, for (a) we see that the mean number of
  iterations are about the same when weighting by \(\Sigma_n\) and
  \(\sigma_n\) (maybe maximum number of iterations for \(\sigma_n\) is
  higher, at least for some sample sizes), whereas unweighted one-step
  requires many more iterations.
\item For (b) (many time-points, but only few repetitions),
  \(\Sigma_n\) converges much faster than \(\sigma_n\).
\item For (d) (competing risks), the norm weighted by \(\sigma_n\)
  requires in some situations many more iterations than \(\Sigma_n\).
\item For (d) (competing risks), where there is one late time-point,
  \(\Sigma_n\) seems to win in terms of relative mse.
\item A collection of some issues:
  \begin{itemize}
  \item Algorithms cannot handle too many time-points combined with
    too small sample sizes; fails to give results. \textit{I have not
      investigated this further}.
  \item \(\Sigma_n\) is not always invertible; I suppose this may be
    due to the choice of time-points and what is observed in the
    data. This is better when sample size \(n\) is
    larger. \textit{Could investigate this further}.
  \end{itemize}
\end{itemize}

\newpage\phantom{blabla}
\newpage\phantom{blabla}




\subsection{Exploratory results (old)}
\label{sec:empirical:results:old}

The following plots show some results from some of the simulations I
have run, in order to:
\begin{itemize}
\item Investigate the impact of varying the norms of Example
  \ref{ex:1}--\ref{ex:3} (see Section \ref{sec:hilbert:space:norms}):
  Figures \ref{fig:plot:study1} and  \ref{fig:plot:surv:study2}.
\item Verify the theoretical properties when targeting three
  cause-specific average treatment effects simultaneously across
  multiple time-points: Figure \ref{fig:plot:study2}.
\item Illustrate the use of one-step TMLE on just a simple dataset:
  Figure \ref{fig:plot:one1}.
\item Illustrate the effect of the targeting step in the survival
  setting to remove bias: Figure \ref{fig:plot:surv:study3}.
\end{itemize}

\newpage 

% \subsection{Simulation study}


\begin{figure*}[!h] % h: here, t: top, b: bund, !: ikke p�nt
  \centering
  \includegraphics[width=1.0\textwidth,angle=0]{./figures/plot-results-3-tau035-tau05-n1000-M500.pdf}
  \caption{Results from targeting the average treatment effect on
    three cause-specific risk functions across two time-points
    simultaneously; each row corresponds to the three examples of
    Hilbert space norms, see Section \ref{sec:hilbert:space:norms}.
    Below and above the points, respectively, are shown the coverage
    of confidence intervals across simulation repetitions and the MSE
    relative to the corresponding nonparametric Aalen-Johansen. The
    points mark the average of TMLE estimates across simulation
    repetitions, and the true values are shown as the dashed red
    line. }
  \label{fig:plot:study1}   
\end{figure*}

\begin{figure*}[!h] % h: here, t: top, b: bund, !: ikke p�nt
  \centering
  \includegraphics[width=1.0\textwidth,angle=0]{./figures/plot-results-3-tau01-tau0225-tau035-tau0475-tau06-n1000-M500.pdf}
  \caption{Results from targeting the average treatment effect on
    three cause-specific risk functions across multiple time-points
    simultaneously. Below and above the points, respectively, are
    shown the coverage of confidence intervals across simulation
    repetitions and the MSE relative to the corresponding
    nonparametric Aalen-Johansen. The points mark the average of TMLE
    estimates across simulation repetitions, and the true values are
    shown as the dashed red line.  }
  \label{fig:plot:study2}   
\end{figure*}

% \subsection{Illustration on a single dataset}



\begin{figure*}[!h] % h: here, t: top, b: bund, !: ikke p�nt
  \centering
  \includegraphics[width=1.0\textwidth,angle=0]{./figures/plot-illustration-tau01-tau0156-tau0211-tau0267-tau0322-tau0378-tau0433-tau0489-tau0544-tau06-n1000-with-true-curves.pdf}
  \caption{Results from targeting three cause-specific and
    treatment-specific risk functions across multiple time-points
    simultaneously for a single dataset. The dashed lines show the
    true curves.  }
  \label{fig:plot:one1}   
\end{figure*}

\begin{figure*}[!h] % h: here, t: top, b: bund, !: ikke p�nt
  \centering
  \includegraphics[width=1.0\textwidth,angle=0]{./figures/plot-results-survival-tau01-tau0256-tau0411-tau0567-tau0722-tau0878-tau1033-tau1189-tau1344-tau15-n1000-M500.pdf}
  \caption{Results from targeting the average treatment effect on
    survival across multiple time-points simultaneously. Below and
    above the points, respectively, are shown the coverage of
    confidence intervals across simulation repetitions and the MSE
    relative to the corresponding nonparametric Kaplan-Meier
    estimator. The points mark the average of TMLE estimates across
    simulation repetitions, and the true values are shown as the
    dashed red line.  }
  \label{fig:plot:surv:study2}   
\end{figure*}


\begin{figure*}[!h] % h: here, t: top, b: bund, !: ikke p�nt
  \centering
  \includegraphics[width=1.0\textwidth,angle=0]{./figures/plot-results-misspecified-survival-tau01-tau0256-tau0411-tau0567-tau0722-tau0878-tau1033-tau1189-tau1344-tau15-n1000-M500.pdf}
  \caption{Results from targeting the average treatment effect on
    survival across multiple time-points simultaneously.  The points
    mark the average of TMLE estimates across simulation repetitions,
    and the true curve is shown as the dashed red line; the upper plot
    shows the curve estimated with a misspecified Cox model, and the
    lower plot show shows the curve resulting from adding the TMLE
    step to the misspecified Cox model. }
  \label{fig:plot:surv:study3}   
\end{figure*}

%\section*{Practical considerations}

%When sample size is small, we expect that we cannot solve all
%equation. Likewise, when there are too may causes.


%\subsection*{Illustration with one dataset}
